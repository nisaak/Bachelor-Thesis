% !TeX spellcheck = en_GB
% !TeX encoding = UTF-8
% !TeX root = ../report.tex

\chapter{Introduction}
\label{chp:Introduction}

%keywords 
%drive by wire, components, idea, motivation, objective
%many different solutions and approaches possible

%context of project (go more into detail)
The purpose of this work was to design a drive-by-wire system for future use in the field of autonomous self driving cars. The idea of the overall project is to have a fleet of autonomous vehicles, mainly to test fleet management algorithms. %correct??
This go-kart prototype was made as a proof of concept and for early testing of the software and hardware.

%goal and reason for necessity
Our objective therefore was plan and realise a solution for every critical component, i.e. throttle, brake and steering. In order to test A.I. and the fleet management algorithms, a stable and reliable low level platform needs to be in place. Requirements for the indiviual components in the drive-by-wire system were i.e. precision, speed, reliable communication and flexibility. Planing included steps such as finding the parts, how to build it into the kart, communicating with the part and supplying the right power.
%state of the art (different solutions, challenges,why this work and that not etc.)
Designing a drive-by-wire system poses many challenges, which can be approach in many different ways. Most drive-by-wire vehicles are being conceptualised with safety as their highest priority. This is because those vehicles transport passengers and their well-being is of utmost importance. That is why these system contain many fallback levels, which are rather difficult to realise. Our kart does not contain such fallback levels, simply because it is self-driving and the drive-by-wire system will not be active when a human is at the wheel. 

%shortly describe chapters and their importance,connection, and specific order.

