% !TeX spellcheck = en_GB
% !TeX encoding = UTF-8
% !TeX root = ../report.tex

\chapter{Communication}
\label{chp:Communication}

To handle the communication between all devices, a fast and easy to implement data exchange method was required. RIMO, as well as the other manufacturers of our components made use of the standardized industrial application CANOpen. Because CANOpen is based on CAN, we will first describe the CAN bus protocol.

\section{CAN Basics}


CAN stands for Control Area Network and consists of two main layers, namely the physical layer and the data link layer. CAN was developed in 1986 and is used for data exchange between different stations in a network, based on serial communication. Messages are received and transmitted via broadcasting, making every message available for all of the connected stations. 
%importance and pros cons of can

The rise in electronification in many parts of the industry called for simpler and more efficient means of communication. Extensive wiring still resulted in rather limited data exchange. A way out of this was presented by serial bit data exchange and connecting up all electronic control units to a single bus. Depending on the bus length, CAN offers up to 1 Mbit/s of data rate, while remaining robust and reliable, even in a noisy environment.

The physical layer consists of a twisted pair of wires, which can be shielded if required. The value of a bit was determined by the difference in voltage of the two wires. If the voltages are the same, the bit is recessive. If the difference is higher than 0.9 V, the bit is dominant. As both wires are affected by the same electromagnetic disturbances, their difference in voltage will not vary. The data link is therefore immune to electromagnetic disturbances. By twisting the wires, the magnetic field generated will also be reduced significantly.

%graphic of can network
%elaborate on physical layer and simple message, 60 ohm when measured as both ends parallel
%baud rate

To reduce reflections in the data cables with rates higher than 125kbit/s, it is recommended to terminate the ends of the communication lines with termination resistors with 120 Ohm. 

Each CAN message contains the following structure.
%picture and example of can message

Especially the role of the identifier bit is important, as it handles and represents the message priority.

\pagebreak

\section{CANOpen}

CANOpen is a higher-layer protocol based on the aforementioned CAN. In the following, we will provide information on the project-relevant aspects of CANOpen.
All real-time data is exchanged via process data objects (PDO). The adress of each individual PDO can be found in a standardized table by CAN in Automation (CiA). PDO's contain data from a single or different objects from the object dictionary. This maps each bit of the data section of a PDO to a certain object. 
To change PDO mapping or customise parameters in the object dictionary, service data objects (SDO) are used. These however do not transmit and receive real-time data, but are only used for service purposes. They usually contain the index, subindex and the new value of the respective object.

Each CAN device has a personal node, ranging from 1 to 127. This ensures that every message reaches it's corresponding device.
% example of pdo can message

% example of sdo can message
The standard table of CAN-Id's can be found below.
\\

%keywords
%transmission type, can id, sdo, pdo
There are various transmission types avaible in CANOpen for PDOs. Some are manufacturer specific, therefore only the types relevant for our project will shortly be described.

Transmission type 0 (acyclic synchronous)

%write more about this 

Transmission type 1-240 (cyclic synchronous)

This transmission type relies on sync messages. After every n sync messages, the PDO transmits it's data. So for example, if the transmission type is 6, the PDO will send after every 6th sync message.

Transmission type 254/255 (asynchronous)

These transmission types are event-triggered, where the event is manufacturer specific for 254 and for 255 defined in the CANOpen device profile.


%busheavy, buslight, busoff, correct baudrate etc.

% Add table of CAN ID
\begin{tabular}{|c|c|c|}
	\hline 
	\textbf{COB} & \textbf{Function Code} & \textbf{Resulting CAN-ID} \\ 
	\hline 
	NMT &  0000$_{b}$  & $0 (000_{h})$  \\ 
	\hline 
	CODE & $0001_{b}$ &  $128 (080_{h})$\\ 
	\hline 
	TIME & $0010_{b}$ &  $256 (100_{h})$\\ 
	\hline 
	EMCY & $0001_{b}$ &   $129 (081_{h}) $–  255 (0FF$_{h})$ \\ 
	\hline 
	PDO1(tx) & $0011_{b}$ &  $385 (181_{h}) $ – 511 (1FF$_{h})$\\ 
	\hline 
	PDO1(rx) & $0100_{b}$ &  $513 (201_{h}) $ –  639 (27F$_{h})$\\ 
	\hline 
	PDO2(tx) & $0101_{b}$ &  $641 (281_{h}) $ –  767 (2FF$_{h})$\\ 
	\hline 
	PDO2(rx) & $0110_{b}$ &  $769 (301_{h})$  –  895 (37F$_{h})$\\ 
	\hline 
	PDO3(tx) & $0111_{b}$ &  $897 (381_{h}) $ – 1023 (3FF$_{h})$\\ 
	\hline 
	PDO3(rx) & $1000_{b}$ &  $1025 (401_{h}) $ –  1151 (47F$_{h})$\\ 
	\hline 
	PDO4(tx) & $1001_{b}$ &  $1153 (481_{h}) $ –  1279 (4FF$_{h})$\\ 
	\hline 
	PDO4(rx) & $1010_{b}$ & $1281 (501_{h}) $ –  1407 (57F$_{h})$\\ 
	\hline 
	SDO(tx) & $1011_{b}$ & $ 1409 (581_{h}) $ –  1535 (5FF$_{h})$\\ 
	\hline 
	SDO(rx) & $1100_{b}$ & $1537 (601_{h}) $ –  1663 (67F$_{h})$\\ 
	\hline 
	NMT error control &$ 1110_{b} $& $ 1793 (701_{h}) $ –  1919 (77F$_{h})$\\ 
	\hline 
\end{tabular} 
\\
\\
\textit{Table 1, COBs and there corresponding CAN-IDs}
%adjust table number







%brief explaination on additional features, their advantages and how they benefited us.
\subsection{Implementation}
%explain what commands we used, object dictionary and relevant commands/messages.


\subsubsection{ACD}

%write about RPDO1 RPDO2 conflict, our assumptions what if couldve been, namely phyiscal connection, sync messages, overload, messages overlapping etc. solved by disabling rpdo2

In order to communicate with the ACD, the CAN Node-Id of each ACD had to be determined. One of Rimo's pdf files, namely "Setting up ACD Controller \& Connection Diagram", serves exactly this purpose. 
%refer to appendix witht file.
To find the respective node-id's, we had to check if either pin 12 (DI5) or pin 20 (DI6) was connected to any other pins on the ACD 4805 K1 connector. In our case, pin 12 of the right ACD was connected to pin 1, making it node 6. Pin 12 of the left ACD was not connected and therefore making it node 5. Afterwards our findings were confirmed when we connected the go-kart to our CAN bus. With the help of a CAN-USB adapter, we were able to receive and send messages, and after a while controlling the kart.
Because the go-kart does not communicate via CAN by default but only for service and remote control purposes, we had to put all nodes into operational mode by sending a NMT start messages to all nodes, namely node 5 and 6.
The ID of a NMT message is 0x000, to start the nodes the instruction code 0x01 had to be used. To reach all nodes simultaneously, the node adress needed to be 0x0. 

Issues with CANOpen and dSpace. 

When we tried to control the ACD via CANOpen with the Microautobox, the communication presented itself to be an issue. The wheels were not turning smooth, but rather interrupted from time to time. It was clear that something had to be disrupting the communication.
With the help of a CAN-USB adapter, we were able to monitor the CAN bus. After checking all physical layers, we came to the conclusion that it must be a software problem. Our first approach was to change synchronisation times. This however did not have any effect, neither did changing the baud rate or adjusting the step size for Matlab's solver.



We established connection between MABX and a receiver, in our case a laptop connected to the MABX CAN via CAN USB.


Two's complement is convenient way to store integers, such that adding and subtracting with negative numbers becomes very easy. This was used on the ACD 4805, where certain values, such as the rotational speed of the wheels, can be negative. 
The basic principles of two's complement are the following.

- Zero is represented by all 0's. 
e.g. 0 0 0 0 = 0

- The maximum positive integer is $ 2^{number of bits-1}-1 $.
So for 4 bits, the biggest integer is therefore 0 1 1 1 = 7, and not 1 1 1 1 = 15 as in the standard notation.

- if the integer is negative, 1's and 0's switch roles, starting from all one's, e.g. 1 1 1 1 = -1. This increases the range for negative numbers by one.

So for 4 bits it looks as follows.

\begin{tabular}{llll}
0 0 0 0 = 0 & 0 1 0 0 = 4 & 1 1 1 1 = -1 & 1 0 1 1 = -5\\
0 0 0 1 = 1 & 0 1 0 1 = 5 & 1 1 1 0 = -2 & 1 0 1 0 = -6\\
0 0 1 0 = 2 & 0 1 1 0 = 6 & 1 1 0 1 = -3 & 1 0 0 1 = -7\\
0 0 1 1 = 3 & 0 1 1 1 = 7 & 1 1 0 0 = -4 & 1 0 0 0 = -8
\end{tabular}

So an easy way to find the negative integer of a positive integer is to convert the desired decimal number to binary, inverting all 0's and 1's and then adding 1. A more hands on approach is to again convert to binary, starting from the right to find the first 1 and inverting all bits to the left of it.

So in order to have a rotational speed of -500 revolutions per minute, the following steps have to be taken for a signed 16 bit integer.

500 = 0 0 0 0 \: 0 0 0 1 \: 1 1 1 1 \: 0 1 0 0

Now invert all bits.

1 1 1 1 \: 1 1 1 0 \: 0 0 0 0 \: 1 0 1 1

And add 1

1 1 1 1 \: 1 1 1 0 \: 0 0 0 0 \: 1 1 0 0 = -500

The second method would result in the following steps.

Starting from the right, find first 1.

500 = 0 0 0 0 \: 0 0 0 1 \: 1 1 1 1 \: 0 \textbf{1} 0 0

Invert all consecutive bits to the left of it.

1 1 1 1 \: 1 1 1 0 \: 0 0 0 0 \: 1 \textbf{1} 0 0 = -500

If a signed integer is postive or negative is easy to spot, as it's most significant bit determines if the number is negative or positive. 1 = negative, 0 = positive.

\subsubsection{LinMot}
important settings

For some reason, TPDO3 and TPDO4 would not transmit their data when a sync message was sent. This was tried to solve by setting the intern event timer to around 10 ms and changing the transmission type to 254. However, after a while it would stop transmitting out of a unknown reason. As time was scarce, we simply remapped the pdos, such that the needed data would be transmitted via TPDO2. 

%talk about failsafe for brake

Another issues presented itself while working on the brake. Previously we used a motion command called VAI go to position 16bit, which takes velocity, acceleration/deceleration and position as input and creates a curve for the linear motor, which results in the motion. However, a new command will only be executed, if the value of the motion command count has changed. In the easiest way bit 0 can be toggled. 
To avoid this, we tried a different setting, called PV Stream. This uses a constant stream of position and velocity inputs during a fixed streaming period, interpolates and executes the command. While this seemed very intriguing, it's implementation was not possible. For some reason an error arose, stating that our streaming was too slow. Even after checking the period time with PCAN and checking all setting, the issue could not be resolved.
After that, we went back to the prior way of setting the position.
%%really
%clock not working as it triggers resets all the time