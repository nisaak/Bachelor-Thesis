% !TeX spellcheck = en_GB
% !TeX encoding = UTF-8
% !TeX root = ../report.tex

\chapter{Communication}
\label{chp:Communication}

To handle the communication between all devices, a fast and easy to implement data exchange method was required. RIMO, as well as the other manufacturers of our components made use of the standardized industrial application CANOpen. Because CANOpen is based on CAN, we will first describe the CAN bus protocol.

\section{CAN Basics}


CAN stands for Control Area Network and consists of two main layers, namely the physical layer and the data link layer. CAN was developed in 1986 and is used for data exchange between different stations in a network, based on serial communication. Messages are received and transmitted via broadcasting, making every message available for all of the connected stations. 
%importance and pros cons of can

The rise in electronification in many parts of the industry called for simpler and more efficient means of communication. Extensive wiring still resulted in rather limited data exchange. A way out of this was presented by serial bit data exchange and connecting up all electronic control units to a single bus. Depending on the bus length, CAN offers up to 1 Mbit/s of data rate, while remaining robust and reliable, even in a noisy environment.

The physical layer consists of a twisted pair of wires, which can be shielded if required. The value of a bit was determined by the difference in voltage of the two wires. If the voltages are the same, the bit is recessive. If the difference is higher than 0.9 V, the bit is dominant. As both wires are affected by the same electromagnetic disturbances, their difference in voltage will not vary. The data link is therefore immune to electromagnetic disturbances. By twisting the wires, the magnetic field generated will also be reduced significantly.

%graphic of can network
%elaborate on physical layer and simple message

To reduce reflections in the data cables with rates higher than 125kbit/s, it is recommended to terminate the ends of the communication lines with termination resistors with 120 Ohm. 

Each CAN message contains the following structure.
%picture and example of can message

Especially the role of the identifier bit is important, as it handles and represents the message priority.

\section{CANOpen}

CANOpen is a higher-layer protocol based on the aforementioned CAN. In the following, we will provide information on the project-relevant aspects of CANOpen.




%brief explaination on additional features, their advantages and how they benefited us.
\subsection{Implementation}
%explain what commands we used, object dictionary and relevant commands/messages.

